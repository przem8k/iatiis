\section{Canny Edge Detector}

\paragraph*{}
Having discussed the concept of image gradient and a set of simple fixed-mask gradient operators now we move our attention to the complete step edge detection technique described\cite{Canny86} by Canny, the one-dimensional version of which was our method of choice for detection of step edges in image profiles. 

\paragraph*{}
Let us recall that Canny's method for one dimension may be summarized as follows:
\begin{enumerate}
	\item Compute the derivative of Gaussian of the signal.
	\item Select the resulting points that:
	\begin{enumerate}
		\item Have magnitude above certain minimum value (\textbf{Thresholding})
		\item Have locally maximum magnitude (\textbf{Non-maximum Suppression})
	\end{enumerate}
\end{enumerate}

\paragraph*{}
Canny method for two-dimensional edges retains this schema, but some of 2D-specific details of the individual steps are interesting enough to call for detailed discussion.

\subsection{Gradient by derivative of Gaussian}

\paragraph*{}
The Gaussian function, which we have discussed in the \refchap{1DEdgeDetection} chapter, has a natural formulation in two dimensions:
\[
    g(x,y)= \frac{1}{2\pi \sigma^2} e^{-\frac{x^2+y^2}{2 \sigma^2}}
\]

\paragraph*{}
Apart from the optimal\footnote{Under the criteria assumed by Canny.} trade-off between noise attenuation and edge preservation, the two-dimensional Gaussian filter has two useful, 2D-specific features. The first one is \textbf{isotropy} - the response of the filter depends only on the distance from $(x,y)$ to $(0,0)$; thus the smoothing does not introduce any bias dependent on the direction of the image edges. 

\paragraph*{}
The second interesting feature is \textbf{separability}, which means that the application of the two-dimensional Gaussian filter is equivalent to successive application of two one-dimensional operators, each processing the signal in different dimension. Indeed:

\begin{align*}
    g_{\sigma}(x,y) &= \frac{1}{2\pi \sigma^2} e^{-\frac{x^2+y^2}{2 \sigma^2}} \\
    				&= \frac{1}{\sqrt{2\pi \sigma^2}} e^{-\frac{x^2}{2 \sigma^2}} \cdot \frac{1}{\sqrt{2\pi \sigma^2}} e^{-\frac{y^2}{2 \sigma^2}} \\
    				&= g_{\sigma}(x)\cdot g_{\sigma}(y)
\end{align*}

\paragraph*{}
Therefore to perform the Gaussian smoothing on an image, we may simply apply the already discussed one-dimensional smoothing operator successively to each row, and then each column of the image. Gradient extraction based on such smoothing is demonstrated in \reffig{CannyGradient}.

\twoFigures
{img/2DEdgeDetection/tooth_canny_low}
{img/2DEdgeDetection/tooth_canny_high}
{Gradient extraction using Gaussian smoothing with different values of standard deviation ($0.7$ on the left, $2.75$ on the right).}
{CannyGradient}
{\basicWidth}

\paragraph*{}
The example results look promising - in the image on the right the outline of the object is prominent and consistently brighter that any other element of the image; the noise is effectively suppressed. On the other hand, the edges in the image clearly have excessive width, which inhibits the precision of edge localization. Conversely, image on the left exhibits fine edges, but also worse separation of edge and noise intensities. It is worth stressing that accurate adjustment of $\sigma$ is crucial for performance of the algorithm.

\subsection{Hysteresis Thresholding}

\paragraph*{}
Once we have computed the image gradient, we need to threshold the gradient magnitude image selecting the high-gradient, prospective edge locations. Although simple global thresholding described in the \refchap{ImageThresholding} chapter could be employed for that, Canny pointed out an important drawback of this method in the context of edge detection.

\paragraph*{}
Let us assume that we set the minimum magnitude threshold value to $s$, and the image contains an edge mean magnitude of which is near to $s$. Even if the illumination of the scene is uniform, the gradient magnitude of edge pixels is bound to fluctuate around $s$ due to inevitable noise introduced by camera imperfections.

\paragraph*{}
In practice the fluctuation of edge gradient magnitude tend to be quite significant, leading to extraction of fragmented, incomplete edges. \reffig{CannyGlobalThresholding} demonstrates two ineffective global thresholding attempts on example gradient amplitude image. The results on the left are free from noise but incomplete, while lower threshold used to produce the outcome on the right introduces significant amount of noise to the results.

\twoFigures
{img/2DEdgeDetection/tooth_canny_hist_high}
{img/2DEdgeDetection/tooth_canny_low_hist_low}
{Global thresholding of the gradient magnitude image.}
{CannyGlobalThresholding}
{\basicWidth}

\paragraph*{}
Canny proposed\footnote{This approach is a two-dimensional equivalent of the method of proposed\cite{Schmitt38} by Schmitt in the context of electronic trigger, although Canny did not indicate his inspirations on that matter.} to address this issue by setting two thresholds, let us denote them by $l$ and $h$, where $l < h$. The locations of magnitude lower than $l$ are immediately discarded, while locations of magnitude higher than $h$ are immediately accepted. The locations of magnitude between $l$ and $h$ are split into connected components and each component is accepted only if it is adjacent to previously accepted (i.e. of magnitude higher than $h$) locations.

\paragraph*{}
Such topological extension of the usual thresholding, called hysteresis thresholding, proves extremely useful in edge detection applications. \reffig{CannyHysteresisThresholding} demonstrates hysteresis thresholding of the example image using two threshold values used to produce results in \reffig{CannyGlobalThresholding} as $l$ and $h$. As we can see, the edges from the low threshold image are retained in their entirety, while all of the noise is lost.

\oneFigure
{img/2DEdgeDetection/tooth_canny_hist}
{Hysteresis thresholding of the gradient magnitude image.}
{CannyHysteresisThresholding}
{\basicWidth}

\subsection{Non-maximum Suppression and Subpixel Precision}

\paragraph*{}
A direct translation of one-dimensional condition of locally maximal magnitude of edge locations would be to compare each pixel with its neighbors (either 4 or 8), but such approach would effectively yield only isolated, single edge locations and this is clearly not what we want.

\paragraph*{}
To derive correct translation of non-maximum suppression into two dimensions we need to remember that in 1D edge detection we were analyzing profiles perpendicular to actual image edges. As the direction perpendicular to an edge is directly represented by gradient direction, non-maximum suppression in two dimensions should be performed by comparing each pixel magnitude with its two neighbors indicated by the pixel gradient direction.

\paragraph*{}
We may use a similar approach to translate out parabola-fitting technique of acquiring sub-pixel precise positions of edge points to two-dimensions. If we sample the image at three points on the line parallel to gradient direction at a pixel and fit a parabola through the resulting, three-element profile, we obtain a direct equivalent of the fit that we were doing in one-dimensional case.

\begin{refImpl}
Canny edge detection method is implemented by \studio filter \filter{DetectEdges}{2DEdgeDetection}.
\end{refImpl}